% Шаблон (версия от 15.02.2016) предназначен 
% для использования студентами каф. ПМиИ СамГТУ 
% при оформлении отчетов по лабораторным работам. 
% Для настройки пакета listigs использовался материал 
% статьи Михаила Конника aka virens
% <http://mydebianblog.blogspot.ru/2012/12/latex.html>
% Copyright (c) 2016 by Mikhail Saushkin (msaushkin@gmail.com) 
% All rights reserved except the rights granted by the
% Creative Commons Attribution 4.0 International Licence
% <https://creativecommons.org/licenses/by/4.0/>
% Свежая версия шаблона здесь <https://www.overleaf.com/read/sqvxbnhgxxdm>
\documentclass[14pt,a4paper,report]{ncc}
\usepackage[a4paper, mag=1000, left=2.5cm, right=1cm, top=2cm, bottom=2cm, headsep=0.7cm, footskip=1cm]{geometry}
\usepackage[utf8]{inputenc}
\usepackage[english,russian]{babel}
\usepackage{indentfirst}
\usepackage[dvipsnames]{xcolor}
\usepackage[colorlinks]{hyperref}
\usepackage{listings} 
\usepackage{caption}
\usepackage{amssymb}
\usepackage{tcolorbox}
\DeclareCaptionFont{white}{\color{white}} %% это сделает текст заголовка белым
%% код ниже нарисует серую рамочку вокруг заголовка кода.
\usepackage{color} %% это для отображения цвета в коде
\DeclareCaptionFormat{listing}{\colorbox{gray}{\parbox{\textwidth}{#1#2#3}}}
\captionsetup[lstlisting]{format=listing,labelfont=white,textfont=white}
\lstset{% Собственно настройки вида листинга
inputencoding=utf8, extendedchars=\true, keepspaces = true, % поддержка кириллицы и пробелов в комментариях
language=C++,            % выбор языка для подсветки (здесь это Pascal)
numberstyle =\tiny
basicstyle=\small\sffamily, % размер и начертание шрифта для подсветки кода
keywordstyle =\color{ForestGreen},
numbers=left,               % где поставить нумерацию строк (слева\справа)
numberstyle=\tiny,          % размер шрифта для номеров строк
stepnumber=1,               % размер шага между двумя номерами строк
numbersep=5pt,              % как далеко отстоят номера строк от подсвечиваемого кода
backgroundcolor=\color{white}, % цвет фона подсветки - используем \usepackage{color}
showspaces=false,           % показывать или нет пробелы специальными отступами
showstringspaces=false,     % показывать или нет пробелы в строках
showtabs=false,             % показывать или нет табуляцию в строках
frame=single,               % рисовать рамку вокруг кода
tabsize=2,                  % размер табуляции по умолчанию равен 2 пробелам
captionpos=t,               % позиция заголовка вверху [t] или внизу [b] 
breaklines=true,            % автоматически переносить строки (да\нет)
breakatwhitespace=false,    % переносить строки только если есть пробел
escapeinside={\%*}{*)}      % если нужно добавить комментарии в коде
}

\begin{document}
% Переоформление некоторых стандартных названий
%\renewcommand{\chaptername}{Лабораторная работа}
\def\contentsname{Содержание}

% Оформление титульного листа
\begin{titlepage}
\begin{center}
\textsc{ФГОУ ВО Уральский Федеральный Университет \\ имени первого Президента России Б.Н.Ельцина\\[5mm]
Физико-технологический институт\\[2mm]
Кафедра теоретической физики и прикладной математики}

\vfill

\textbf{ОТЧЁТ ПО ЛАБОРАТОРНОЙ РАБОТЕ №1\\[3mm]
«Симуляция двумерной системы частиц взаимодействующих через потенциал Леннарда-Джонса с учетом периодических граничных условий»\\[6mm]
}
\end{center}

\hfill
\begin{minipage}{.5\textwidth}
Студент:\\[2mm] 
Вялова С.А.\\
группа: ФтМ-170403 \\[5mm]

Преподаватель:\\[2mm] 
д.ф.-м.н., профессор\\
Мазуренко Владимир Владимирович\\[5mm]

Консультант:\\[2mm] 
н.с.\\
Сотников Олег Михайлович\\

\end{minipage}%
\vfill
\begin{center}
\today  \\
%\theyear\, г.
 Екатеринбург.
\end{center}
\end{titlepage}

% Содержание
\tableofcontents
\newpage
\chapter{Моделирование двумерной системы частиц, взаимодействующих через потенциал Леннарда-Джонса}
\section{Цель работы}


Разработка программы для симуляции молекулярной динамики двумерной системы частиц, взаимодействующих через потенциал Леннарда-Джонса с учётом периодический граничных условий, а также применение алгоритма Верле для решения уравнения движения. Начальные значения координат и скоростей частиц выбираются так, чтобы выполнялось условие нулевого импульса системы.

\
Реализуются следующие пункты:

\begin{itemize}
\item Возможность задания произвольного числа частиц;
\item расчет кинетической, потенциальной и полной энергии;
\item определение и контроль температуры системы;
\item визуализация всей системы  и траектории отдельной частицы;
\item расчет среднеквадратичного смещения и автокорреляционной функции скорости.
\end{itemize}
\newpage\section{Теоретическая часть }
\subsection{Потенциал Леннарда-Джонса}
%text
Потенциал Леннарда-Джонса представляет собой простую модель парного взаимодействия неполярных молекул, описывающая зависимость энергии взаимодействия двух частиц от расстояния между ними.
Потенциал был предложен Леннардом-Джонсом первоначально для исследования термодинамических свойств инертных газов. Наиболее часто используется так называемый (6-12)-потенциал Леннарда-Джонса, записанный в форме 
\
\begin{equation}
 U = 4 \cdot \varepsilon \cdot [(\sigma/r)^{12} - (\sigma/r)^{6}  ] ,
 \end{equation} 
где $\varepsilon$ - глубина потенциальной ямы, $\sigma$ - значение расстояния между частицами, при котором потенциал равен нулю. Шестая степень убывания отвечает электростатическому диполь-дипольному и дисперсионному притяжению; двенадцатая степень убывания потенциала моделирует достаточно жесткое отталкивание и выбрана из соображений математического удобства.
\

\includegraphics[scale=0.8]{plot_lgpotential}
%\caption{Зависимость температуры системы от шага по времени.}


\subsection{Описание вычислительного метода}
%text
Для моделирования системы взаимодействующих частиц необходимо численно решить классические уравнения движения системы
\begin{equation}
m_{i}\dot{\vec{v_i}}=F_{i}(t, \vec{x_1}, ... ,  \vec{x_N}, \vec{v_1}, ... , \vec{v_N})
\end{equation}
\begin{equation}
\dot{\vec{x_i}}=\vec{v_i}.
\end{equation}
Чтобы начать моделировать двумерную систему взаимодействующих частиц, необходимо задать все координаты частиц, равномерно расположив их относительно друг друга, а также их скорости. Каждая пара частиц взаимодействует через потенциал Леннарда-Джонса.
\begin{equation}
\vec{r}_{n+1} = \vec{r}_{n} + \vec{v}_{n}\cdot{dt}
\end{equation}
\begin{equation}
\vec{v}_{n+1}=\vec{v}_{n}+\frac{1}{2} \cdot (\vec{a}_n+\vec{a}_{n+1})
\end{equation}
Ускорения определяем из потенциала Леннарда-Джонса:
\begin{equation}
\vec{F}_i= -\vec{\nabla}U(\vec{r}) = m_i \vec{a}_i
\end{equation}
Подставляя выражение для потенциала Леннарда-Джонса (1.1) в выражение для силы, действующей на каждую частицу со стороны системы, получаем ускорение i частицы:
\begin{equation}
\vec{a}_i=4  \varepsilon \cdot  \sum\limits_{i=j}^N \sum\limits_{j \neq i}^N{\frac{\vec{r}_{ij}}{m_i | \vec{r}_{ij}|^2}  [2  (\sigma/\vec{r}_{ij})^{12}-(\sigma/\vec{r}_{ij})^6]}
\end{equation}
Из полученных данных возможно вычислить кинетическую (1.8) и потенциальную (1.9) энергии системы:
\begin{equation}
E_{kinetic} = \sum\limits_{i=1}^N{\frac{m_i {\vec{v}}^2_i}{2}}
\end{equation}
\begin{equation}
E_{potential}=4 \varepsilon \sum\limits_{i=1}^N \sum\limits_{s=i+1}^N{ [(\sigma/\vec{r}_{is})^{12} - (\sigma/\vec{r}_{is})^{6}  ]}
\end{equation}
Полная энергия системы (1.10) вычисляется как сумма кинетических и потенциальных энергий всех частиц в системе.
\begin{equation}
E_{total}=E_{kinetic}+E_{potential}
\end{equation}
Важным критерием является неизменность полной энергии системы на каждом шаге.
\ 
Используемый для решение поставленной задачи вычислительный алгоритм требует нулевого суммарного импульса системы частиц.
Требование нулевого суммарного импульса системы удовлетворяется за счёт перенормировки заданных начальных скоростей следующим образом:
\begin{equation}
\vec{v}_{{i} \ new} = \vec{v}_i - {\frac{1}{N}}\sum\limits_{i=1}^N \vec{v}_i
\end{equation}
\newpage\section{Практическая часть}
\subsection{Описание функционала разработанной программы}
%text
Для выполнения поставленной задачи выбран язык программирования C++. Входные данные задаются пользователем с клавиатуры. Программе необходимо передать значение общего числа частиц в системе, значение постоянной решётки, шаг по времени, а также число шагов по времени, значения параметров модели $\sigma$ и $\varepsilon$.
\

Далее программа генерирует произвольные значения исходных координат частиц в системе, произвольные значения начальных скоростей частиц в указанных линейных диапазонах, впоследствии пересчитывая скорости в соответствии с требованием нулевого суммарного имульса системы. Вычисляются ускорения и скорости для следующего шага по времени, координаты частиц в системе с учётом периодических граничных условий, и далее, в цикле по шагу по времени, программа вычисляет ускорения для новых положений частиц, скорости, кинетическую, потенциальную энергии системы.
\

\begin{figure}[h]
\center{\includegraphics[width=1\linewidth]{program}}
\caption{Демонстрация выполняющейся программы.}
\label{ris:image}
\end{figure}
Запуск программы осуществляется из коммандной оболочки bash (рисунок 1.1). В ходе выполнения программа для каждого шага по времени выводит в терминал значения кинетической, потенциальной и полной энергии системы, температуру системы, а также значение суммарного импульса для контроля, а также в выходные файлы на каждом шаге записываются координаты частиц, кинетическая, потенциальная и полная энергии системы, температура системы, значение функции среднеквадратического смещения частиц и автокорреляционной функции скорости. По записанным данным могут быть построены графики зависимостей данных величин от шага по времени.
\



\
\subsection{Визуализация всей системы и траектории отдельной частицы}
Заданное число частиц системы располагается в виде равномерной квадратной сетки внутри ячейки со стороной, равной постоянной решетки. Функционал разработанной программы для моделирования данной системы частиц предполагает файл, содержащий набор всех координат частиц на каждом шаге моделирования, поэтому эволюция положения частиц системы была визуализирована с использованием пакета Gnuplot в виде gif-изображения. Положения всех частиц системы, находящейся в исходном состоянии и на некотором шаге в ходе моделирования, представлены на рисунке 1.2(а) и рисунке 1.2(б) соответственно.


\begin{figure}[]
\begin{minipage}[]{0.5\linewidth}
\center{\includegraphics[width=1\linewidth]{initial} \\ а)}
\end{minipage}
\hfill
\begin{minipage}[]{0.5\linewidth}
\center{\includegraphics[width=1\linewidth]{system} \\ б)}
\end{minipage}
\caption{Положение частиц в системе а) в исходном состоянии; б) в ходе моделирования.}
\end{figure}

В ходе моделирования были получены графики зависимости кинетической, потенциальной и полной энергии системы от шага по времени (рисунок 1.3), а также температуры системы и средней температуры системы от шага по времени (рисунок 1.4).
\
\begin{figure}
\includegraphics[scale=0.66]{plot_energy}
\caption{Зависимость энергии системы от шага по времени.}
\end{figure}

\begin{figure}
\includegraphics[scale=0.66]{plot_temperature}
\caption{Зависимость температуры системы от шага по времени.}
\end{figure}


\subsection{Выводы}

\end{document}
